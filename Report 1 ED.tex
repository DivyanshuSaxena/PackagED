%%%%%%%%%%%%%%%%%%%%%%%%%%%%%%%%%%%%%%%%%%%%%%%%%%%%%%%%%%%%%%%%%%%%%
% LaTeX Template: Project Titlepage Modified (v 0.1) by rcx
%
% Original Source: http://www.howtotex.com
% Date: February 2014
% 
% This is a title page template which be used for articles & reports.
% 
% This is the modified version of the original Latex template from
% aforementioned website.
% 
%%%%%%%%%%%%%%%%%%%%%%%%%%%%%%%%%%%%%%%%%%%%%%%%%%%%%%%%%%%%%%%%%%%%%%

\documentclass[12pt]{report}
\usepackage[a4paper]{geometry}
\usepackage[myheadings]{fullpage}
\usepackage{fancyhdr}
\usepackage{lastpage}
\usepackage{graphicx, wrapfig, subcaption, setspace, booktabs}
\usepackage[T1]{fontenc}
\usepackage[font=small, labelfont=bf]{caption}
\usepackage{fourier}
\usepackage[protrusion=true, expansion=true]{microtype}
\usepackage[english]{babel}
\usepackage{sectsty}
\usepackage{url, lipsum}
\usepackage[inline]{enumitem}
\usepackage{amsmath}
\usepackage{setspace}
\usepackage{hyperref}
\doublespacing
\renewcommand{\arraystretch}{0.8}

\newcommand{\HRule}[1]{\rule{\linewidth}{#1}}
\onehalfspacing
\setcounter{tocdepth}{5}
\setcounter{secnumdepth}{5}

%-------------------------------------------------------------------------------
% HEADER & FOOTER
%-------------------------------------------------------------------------------
\pagestyle{fancy}
\fancyhf{}
\setlength\headheight{15pt}
\fancyhead[L]{Design Practices}
\fancyhead[R]{IIT DELHI}
\fancyfoot[R]{Page \thepage\ of \pageref{LastPage}}
%-------------------------------------------------------------------------------
% TITLE PAGE
%-------------------------------------------------------------------------------

\begin{document}

\title{ \normalfont  \textsc{Department of Computer Science and Engineering} \\[2pt]
		\normalfont  {IIT Delhi}
		\\ [2.0cm]
		\HRule{0.5pt} \\
		\LARGE \textbf{\uppercase{Mathematical Model of a Software Package for Engineering Drawing }}
		\HRule{2pt} \\ [0.5cm]
		\normalsize \today \vspace*{5\baselineskip}}

\date{}

\author{
		Divyanshu Saxena | Akshat Khare \\ 
		Design Practices | COP 290   \\
		 }

\maketitle
\tableofcontents
\newpage

%-------------------------------------------------------------------------------
% Section title formatting
\sectionfont{\scshape}
%-------------------------------------------------------------------------------

%-------------------------------------------------------------------------------
% BODY
%-------------------------------------------------------------------------------

\section*{Introduction}
\addcontentsline{toc}{section}{Introduction}
% \lipsum
We have to design and implement a software package for Engineering drawing. The package should have the following functionalities:
\begin{enumerate}
\item We should be able to interactively input or read from a file either i) an isometric drawing and a 3D object model or ii) projections on to any cross section.
\item Given the 3D model description we should be able to generate projections on to any cross section or cutting plane.
\item Given two or more projections we should be able to interactively recover the 3D description and produce an isometric drawing from any view direction.
\end{enumerate}
For the Step 1 of this assignment we are supposed to work out a mathematical model for the problem.We have figure out number of views that are necessary and sufficient, process to compute projections given the 3D description, process to compute the 3D description given one or more projections and number of interactions that are necessary.

To make the problem simpler we are only consider polyhedral object, i.e., there would be no curvilinear planes in the 3d object.





%-------------------------------------------------------------------------------
% REFERENCES
%-------------------------------------------------------------------------------
\newpage
\section*{Abstract}
\addcontentsline{toc}{section}{Abstract}
The underlying requirement for performing this project was to develop an algorithm by which we can make the computer understand the object descriptions and by using this understanding of the machine we can process the data for developing tools to give functionalities, such as providing 3D description from orthographic views and vice versa.

This report gives an appropriate mathematical model for the purpose as mentioned in the Introduction. This project is about developing the basic building block to develop that multi-functional tool.

A major aspect of this project also includes to test the correctness of the well known established algorithms of the branch of engineering, Engineering Drawing, thereby, proving or disproving them.

This project is also the first hand experience of ours to develop a package capable of widespread use.
\newpage
\section*{Specifications}
\addcontentsline{toc}{section}{Specifications}
We have divided the specifications of our model into two types:
\begin{enumerate}
\item 3D Object Description to 2D Projection on a Plane
\item 2D Orthographic Projections to 3D Object Description
\end{enumerate}
We have done so because the information required and delivered is different in each case and are dealing with both cases separately.

\subsection*{3D Object Description to 2D Projection on a Plane}
\addcontentsline{toc}{subsection}{3D Object Description to 2D Projection on a Plane}
We will require these three set of informations for getting the information about the 3d object:
\begin{enumerate}
\item Set of labeled vertices as tuples of x, y and z coordinates.
\item Set of edges as pairs of labels of vertices connecting the extreme ends.
\item Set of planes as tuples of labels of vertices bounding that plane.
\end{enumerate}
We will deliver the views as represented by three sets of information about the 2d representations:
\begin{enumerate}
\item Set of tuple of points labeled according to order of their increasing distance from the plane as tuples of x,y,z depending on the view of consideration.
\item Set of visible edges as pairs of vertices connecting the extreme ends, labeled in order of their increasing distance from the plane. 
\item Set of hidden edges as pairs of vertices connecting the extreme ends, labeled in order of their increasing distance from the plane.
\end{enumerate}


\subsection*{2D Projections to 3D Object Description}
\addcontentsline{toc}{subsection}{2D Projections to 3D Object Description}
We will develop the 3d model based on two views(orthographic to each other) that are given by the user in two sets of information:
\begin{enumerate}
\item Set of points as tuples of x,y,z depending on the view of consideration. (Note: A single coordinate tuple may be labeled more than once according to order of the increasing distance from the projection plane.)
\item Set of edges as pairs of vertices connecting the extreme ends.
\end{enumerate}
Each view shall require the above two sets to be specified to our mathematical model.

We will deliver the 3D object as represented by two sets of information about the 3d representations:
\begin{enumerate}
\item Set of labeled vertices as tuples of x, y and z coordinates.
\item Set of edges as pairs of labels of vertices connecting the extreme ends.
\end{enumerate}
We plan to represent the data taken and given to user in the form of JSON objects or matrices in our software package because this kind of data representation is widely accepted and used. Presently, in our mathematical model, we'll use matrices to represent all sorts of informations unless otherwise specified.
% End of specifications
\newpage
\section*{Mathematical Model}
\addcontentsline{toc}{section}{Mathematical Model}
Based on our Specifications, we develop two models to represent and formulate our descriptions from the given input. Here are the two mathematical models:
\begin{enumerate}
\item 3D Object Description to 2D Projection on a Plane
\item 2D Orthographic Projections to 3D Object Description
\end{enumerate}
\subsection*{3D Object Description to 2D Projection on a Plane}
\addcontentsline{toc}{subsection}{3D Object Description to 2D Projection on a Plane}
% 3d to 2d content
% ---------------------------
% This the 3d to 2d
% ---------------------------
Let us consider the input from the user as:
\begin{itemize}[nolistsep,noitemsep]
\item The set \(\mathcal{A} = \{A_{i}\} \) of vertices with labels as \(A_{i}\) each of which is represented as a column matrix containing three elements: \(\begin{bmatrix} x_{i} \\ y_{i} \\ z_{i} \\ \end{bmatrix}\)  
\item The set \(\mathcal{E} = \{E_{j}\} \) of edges with each element representing an edge between two vertices as: \(E_{j} = (A_{i},A_{k}) \)
\item The set \(\mathcal{F} = \{F_{i}\} \) of faces with each element as a tuple of vertices forming a face in the 3D object, such that \(F_{i} = (A_{i_{1}},A_{i_{2}}, \dots, A_{i_{k}}) \)
\item The projection plane on which the projection is to be taken, \(\mathcal{P} = \begin{bmatrix}
a \\ b \\ c \\ d \end{bmatrix} \)
\end{itemize}
We begin the description of our projection by first, projecting each point \(A_{i} \) in \(\mathcal{A} \) on the projection plane \(\mathcal{P} \). \\
For this we use \textbf{Routine A} (mentioned later) to evaluate the projections of each of the points. We form the set \(\mathcal{A'} = \{A_{i}'\} \), the set of the projected vertices on the plane \(\mathcal{P} \). \\
Once, the set of projected vertices is found, we can correspond each edge \(E_{j}\) to \(E_{j}'\), the corresponding projected edge, as: \(E_{j}' = (A_{i}',A_{k}') \); thereby, constituting the set \( \mathcal{E'} = \{E_{k}'\} \) of projected edges onto the plane of projection \(\mathcal{P} \). \\
The above procedure gives us a raw two dimensional projection of the 3D object on our projection plane. Next, we aim to define the hidden points and edges in this projection. The procedure to determine hidden vertices is shown in formulated in \textbf{Routine B} (mentioned later). Later, we use \textbf{Routine C} to determine the hidden edges in our 2D projection.

\subsubsection*{Routine A}
\addcontentsline{toc}{subsubsection}{Routine A: Projection of a point on a plane}
Let's consider a vertex \(A_{i} = \begin{bmatrix} x_{i} \\ y_{i} \\ z_{i} \\ \end{bmatrix}\) and a plane \(\mathcal{P} \), whose equation is represented as: \\ 
\(\begin{bmatrix} a \\ b \\ c \end{bmatrix}^{T} \begin{bmatrix} x \\ y \\ z \end{bmatrix} = \begin{bmatrix} d \end{bmatrix}\). Now, a general point on the perpendicular from \(A_{i} \) to \(\mathcal{P} \) can be represented as: \(\begin{bmatrix}  x_{i} \\ y_{i} \\ z_{i} \\ \end{bmatrix} + t \begin{bmatrix} a \\ b \\ c \end{bmatrix} \). The solution of this point with the plane \(\mathcal{P} \) shall be the projection point of the vertex \(A_{i} \), which can be evaluated using the equation: \\
\begin{center}
\boxed{
\begin{bmatrix} a \\ b \\ c \end{bmatrix}^{T} \Bigg (\begin{bmatrix}  x_{i} \\ y_{i} \\ z_{i} \\ \end{bmatrix} + t \begin{bmatrix} a \\ b \\ c \end{bmatrix} \Bigg) = \begin{bmatrix} d \end{bmatrix}
}
\end{center}
The solution to the above equation gives t which can be used to evaluate projected point \(
A_{i}' \) for the vertex \(A_{i}\).

\subsubsection*{Routine B: Procedure to determine hidden points in the projection of a 3D pbject on a projection plane}
\addcontentsline{toc}{subsubsection}{Routine B: Procedure to determine hidden points in the projection of a 3D pbject on a projection plane}
Let us take the projected point \(A_{i}' \) of the vertex \(A_{i} \). Next, consider the set \(\mathcal{F}. \forall F_{i}' \in \mathcal{F'}\) (the corresponding faces on the projected plane), we check if \(A_{i}'\) lies in the interior of \(F_{i}'\), using the \textbf{Ray Casting Algorithm}, described in \textbf{Subroutine D} (mentioned later).
\\
\(A_{i}\) shall be a hidden point iff it is shadowed by a face (including its boundaries), for which the projection \(A_{i}'\) must lie on or within the projected face \(F_{k}'\) and the points \(A_{i}\) and \(A_{i}'\) must be on opposite sides of plane containing the face \(F_{k}\) (Refer \textbf{Subroutine E} for details). \\
This determines whether the point \(A_{i} \) is hidden or not.

\subsubsection*{Routine C: Procedure to determine hidden edges in the projection of a 3D object}
\addcontentsline{toc}{subsubsection}{Routine C: Procedure to determine hidden edges in the projection of a 3D object}
Consider the edge \(E_{j} = (A_{i},A_{k}) \) and the corresponding projection \(E_{j}' = (A_{i}',A_{k}') \) on the plane \(\mathcal{P} \). Next, we determine if the points \(A_{i}\) and \(A_{k}\) are hidden or not. If both the points are hidden, then \(E_{j}\) must definitely be hidden. \\
For other cases, we evaluate the following: \(\forall F_{j} \in \mathcal{F}\), evaluate if the point \(A_{i}'\) is in the interior of the polygon described by \(F_{j}'\). This can be evaluated using the formulation in \textbf{Routine B}. If \(A_{i}'\) is in the exterior of the polygon described by \(F_{j}'\), we move on to the next face \(F_{j+1}' \in \mathcal{F}\), else we have the following cases: 
\begin{itemize}[nolistsep,noitemsep]
\item If the point \(A_{i}\) is behind the face \(F_{j}\), we evaluate the point of intersection of the edge \(E_{j}'\) with the polygon described by the face \(F_{j}'\) in the 2D projection. \\
We take the first point of intersection of the edge \(E_{j}'\) with the face \(F_{j}'\). Let's call this point \(B_{i}'\). The line segment \(A_{i}\)\(B_{i}\) shall be hidden, where \(B_{i}\) is the corresponding point of \(B_{i}'\) on the 3D object. \\
To evaluate the remaining part \(B_{i}\)\(A_{k}\), we follow Routine C for this dummy edge \(E_{j}'' = (B_{i},A_{k}) \). We evaluate \(B_{i}\) using the following formulation: \\
\begin{center}
\centering
\( B_{i} = \begin{bmatrix}  x_{b_{i}} \\ y_{b_{i}} \\ z_{b_{i}} \\ \end{bmatrix} = \dfrac{d_{B_{i}'A_{i}'}A_{k} + d_{B_{i}'A_{k}'}A_{i}}{d_{B_{i}'A_{i}'} + d_{B_{i}'A_{k}'}} \) \\
where \( B_{i}' = \begin{bmatrix}  x_{b_{i}}' \\ y_{b_{i}}' \\ z_{b_{i}}' \\ \end{bmatrix}; A_{i} = \begin{bmatrix}  x_{i} \\ y_{i} \\ z_{i} \\ \end{bmatrix}; A_{k} = \begin{bmatrix}  x_{i} \\ y_{i} \\ z_{i} \\ \end{bmatrix}\) and \( d_{XY} = \) distance between points X and Y.
\end{center}
\item If, however, \(A_{i}\) lies on the face \(F_{j}\) then, we evaluate the relative position of the point \(A_{k}\) w.r.t the face \(F_{j}\) using the formulation established in \textbf{Routine B}. Following, we have two cases:
\begin{itemize}[nolistsep,noitemsep]
\item \(A_{k}\) is hidden behind the face \(F_{j}\), in which case the edge \(E_{j} = (A_{i},A_{k}) \) shall be hidden behind the face \(F_{j}\).
\item \(A_{k}\) is on or above the face \(F_{j}\), in which case \(E_{j}\) cannot be hidden by the face \(F_{j}\) and we move to the next face \(F_{j+1} \in \mathcal{F}\).
\end{itemize}
\item If the point \(A_{i}\) is lies ahead of the face \(F_{j}\), then we follow the above procedure with the point \(A_{k}\).
\end{itemize}

\subsubsection*{Subroutine D: Determine if a point lies in the interior of a polygon}
\addcontentsline{toc}{subsubsection}{Subroutine D: Determine if a point lies in the interior of a polygon}
Consider the line \(L_{i}: \begin{bmatrix}  x_{i}' \\ y_{i}' \\ z_{i}' \\ \end{bmatrix} + t \begin{bmatrix} \alpha \\ \beta \\ \gamma \end{bmatrix}; (t \geqslant 0) \) from a point \(A_{i}'\) in a direction defined by the tuple \((\alpha,\beta,\gamma)\), where, \(A_{i}' = \begin{bmatrix}  x_{i}' \\ y_{i}' \\ z_{i}' \\ \end{bmatrix} \) and \(\begin{bmatrix} a \\ b \\ c \end{bmatrix}^{T} \begin{bmatrix} \alpha \\ \beta \\ \gamma \end{bmatrix} = [0] \). \\
Next we consider the points of intersection of this line \(L_{i}\) with the polygon defined by \( F_{i}'\). \( \forall A_{l}' \in F_{i}' \), check if \( A_{l}' = A_{i}'\), in which case the point \(A_{i}' \) is in the interior of \(F_{i}'\), else consider two consecutive vertices \(A_{l}'\) and \(A_{m}' \), which form an edge, say \(E_{k}' = (A_{l}',A_{m}') \) on the face \(F_{i}'\) and check the relative sides of the two points w.r.t. the line \(L_{i}\). \\
If the two points are on the same side, then we do not have a point of intersection of \(L_{i}\) with \(E_{k}'\) else we have one point of intersection if none of the points are on the line \(L_{i}\) and two points of intersection (repeated root) if either of the points is on the line \(L_{i}\). \\
Similarly, we evaluate all points of intersection of the line \(L_{i}\) with the polygon described by \(F_{i}'\) for \(t \geqslant 0 \).\\
If the number of points of intersection of this ray with our polygon is odd, then the point \(A_{i}'\) lies in the interior of the polygon defined by \( F_{i}'\) else it is on the exterior. \\

\subsubsection*{Subroutine E: Relative Position of a Point w.r.t. a Plane}
\addcontentsline{toc}{subsubsection}{Subroutine E: Relative Position of a Point w.r.t. a Plane}
Consider the plane \(P: \begin{bmatrix} a \\ b \\ c \end{bmatrix}^{T} \begin{bmatrix} x \\ y \\ z \end{bmatrix} = \begin{bmatrix} d \end{bmatrix}\) and the points \(A_{i} = \begin{bmatrix}  x_{i} \\ y_{i} \\ z_{i} \\ \end{bmatrix} \) and \(A_{i}' = \begin{bmatrix}  x_{i}' \\ y_{i}' \\ z_{i}' \\ \end{bmatrix} \). The points \(A_{i} \) and \(A_{i}' \) are said to be on the same side of the plane \(P \), iff \\
\centerline{\(d_{A_{i}}.d_{A_{i}'} > 0 \)}
Otherwise, the points \(A_{i} \) and \(A_{i}' \) are said to be on the same side of the plane \(P \), iff \\
\centerline{\(d_{A_{i}}.d_{A_{i}'} < 0 \)}
where, \(d_{A_{i}} =\dfrac{ \begin{vmatrix} \begin{bmatrix} a \\ b \\ c \end{bmatrix}^{T} \begin{bmatrix}  x_{i} \\ y_{i} \\ z_{i} \\ \end{bmatrix} - \begin{bmatrix} d \end{bmatrix} \end{vmatrix} } {\begin{vmatrix} \begin{bmatrix} a \\ b \\ c \end{bmatrix}^{T} \begin{bmatrix}  a \\ b \\ c \\ \end{bmatrix} \end{vmatrix} }\) and \(d_{A_{i}'} =\dfrac{ \begin{vmatrix} \begin{bmatrix} a \\ b \\ c \end{bmatrix}^{T} \begin{bmatrix}  x_{i}' \\ y_{i}' \\ z_{i}' \\ \end{bmatrix} - \begin{bmatrix} d \end{bmatrix} \end{vmatrix} } {\begin{vmatrix} \begin{bmatrix} a \\ b \\ c \end{bmatrix}^{T} \begin{bmatrix}  a \\ b \\ c \\ \end{bmatrix} \end{vmatrix} }\).\\
\textit{(Note: |.| denotes the determinant of matrix and not the absolute value.)}\\


This completes our model and description for the projection of a 3D object on a given projection plane.
\newpage
\subsection*{2D Orthographic Projections to 3D Object Description}
\addcontentsline{toc}{subsection}{2D Orthographic Projections to 3D Object Description}
% 2d to 3d content
The orthographic views given by the user must be on the standard planes of reference. We describe our model, assuming that: 
\begin{enumerate*}
\item Top view is taken on \(x-y\) plane.
\item Front view must be taken on \(x-z\) plane.
\end{enumerate*}. Let us consider the input from the user as:
\begin{itemize}[nolistsep, noitemsep]
\item The set \(\mathcal{T} = \{T_{i}\} \) of lists each of which is a list of points projected on \(x-y\) plane.\(T_{i}\) is therefore a list of points \(A_{T_{j}}\)  each of which is represented as a column matrix containing three elements: \(\begin{bmatrix} x_{T i} \\ y_{Ti}  \\ 0 \\ \end{bmatrix}\). Each \(A_{T_{j}}\) corresponds to the projection of the point \(P_{j}\) on the top view,i.e., \(x-y\) plane. The point in list must me in order of their distance from the plane. Canonically, \(\mathcal{T} = \{[A_{T_{1}},A_{T_{2}},\dots],\dots \} \).
\item The set \(\mathcal{F} = \{F_{i}\} \) of lists each of which is a list of points projected on \(x-z\) plane.\(F_{i}\) is therefore a list of points \(A_{F_{j}}\)  each of which is represented as a column matrix containing three elements: \(\begin{bmatrix} x_{F i} \\ 0  \\ z_{F i} \\ \end{bmatrix}\). Each \(A_{F_{j}}\) corresponds to the projection of the point \(P_{j}\) on the top view,i.e., \(x-z\) plane. The point in list must me in order of their distance from the plane. Canonically, \(\mathcal{F} = \{[A_{F_{1}},A_{F_{2}},..],.. \} \).
\item The set \(\mathcal{E_{T}} = \{E_{T i}\}\) of edges as seen in top view with each element representing an edge between two list of vertices as seen from top view as: \(E_{T j} = (T_{i}, T_{k})\).
\item The set \(\mathcal{E_{F}} = \{E_{F i}\}\) of edges as seen in top view with each element representing an edge between two list of vertices as seen from top view as: \(E_{F j} = (F_{i}, F_{k})\).
\end{itemize} 
We begin the description of our project by first, coinciding the projections of the points from the tuples \(\mathcal{T_{T}}\) and \(\mathcal{T_{F}}\).
For this we use \textbf{Routine F} (mentioned later). After going through this routine we are left with vertices  of the 3d objects: \(\mathcal{V}\) . The process is not yet complete as we have not marked the edges till now. We will now use \textbf{Routine G}(mentioned later too). After going through this routine we have finally marked down the edges too. We are done describing the 3d object now.

\subsubsection*{Routine F: Determination of vertices}
\addcontentsline{toc}{subsubsection}{Routine E: Determination of vertices}
The underlying procedure makes use of the idea that if we draw line perpendicular to the planes of projections passing through the projected points they will coincide at points which correspond to the 3D description of these points. In essence, what is aimed is to evaluate the solution of the lines: \\
\begin{center}
\(L_{1} = \begin{bmatrix}  x_{Ti_{j}} \\ y_{Ti_{j}} \\ z_{Ti_{j}} \\ \end{bmatrix} + t \begin{bmatrix} 0 \\ 0 \\ 1 \end{bmatrix} \space and \space L_{2} = \begin{bmatrix}  x_{Fi_{j}} \\ y_{Fi_{j}} \\ z_{Fi_{j}} \\ \end{bmatrix} + t \begin{bmatrix} 0 \\ 1 \\ 0 \end{bmatrix} \).\\
This procedure evaluates the corresponding 3D point as:
\(\begin{bmatrix}  x_{Ti_{j}} \\ y_{Ti_{j}} \\ z_{Fi_{j}} \end{bmatrix} \). 
\end{center}
Note that: \(x_{Ti_{j}} = x_{Fi_{j}}; y_{Fi_{j}} = 0; z_{Ti_{j}} = 0. \)
\subsubsection*{Routine G: Determination of edges}
\addcontentsline{toc}{subsubsection}{Routine G: Determination of edges}
We start off by determining the possible set of edges from a vertex, by taking the intersection of the connected neighbors (of a point \(A_{T_{j}} \in \mathcal{T} \)) in the top view and front view.\\
\(\forall A_{T_{j}} \in \mathcal{T} \), determine the possible edges \(E_{m}' = (A_{T_{j}},A_{T_{k}}) \) such that \(E_{m} = (P_{j},P_{k}) \) may be a possible edge in the 3D object. Similarly, we determine the possible edges \(E_{m}'' = (A_{F_{j}},A_{F_{l}}) \) To determine the sets \(\{E_{m}'\}\) and \(\{E_{m}''\}\), we follow the following procedure to determine these sets:
\begin{itemize}[nolistsep,noitemsep]
\item  \(\forall A_{T_{k}} \in T_{l}\) other than \(A_{T_{j}}\) itself, \(E_{m}' = (A_{T_{j}},A_{T_{k}})\) must be included in the set of possible edges where \(A_{T_{j}} \in T_{l}\).
\item  \(\forall A_{F_{k}} \in F_{l}\) other than \(A_{F_{j}}\) itself, \(E_{m}'' = (A_{F_{j}},A_{F_{k}})\) must be included in the set of possible edges where \(A_{F_{j}} \in F_{l}\).
\item \(\forall A_{T_{k}} \in T_{n}\) such that there exists an edge \(E_{m}' = (T_{l},T_{n}) \in \mathcal{E_{T}} \), \(E_{m}'\) must be included in set of possible edges. \(A_{T_{j}} \in T_{l}\).
\item \(\forall A_{F_{k}} \in F_{n}\) such that there exists an edge \(E_{m}'' = (F_{l},F_{n}) \in \mathcal{E_{F}} \), \(E_{m}''\) must be included in set of possible edges. \(A_{F_{j}} \in F_{l}\).
\end{itemize}
Now we have the set of possible edges, \(\{E_{m}'\}\) and \(\{E_{m}''\}\). Any edge in the 3D object must reflect in both of the two views in either of the ways mentioned above. The final set of possible edges, containing the point \(A_{T_{j}} \) must, therefore be \(\{E_{m}'\} \cap \{E_{m}''\}\) . This gives us the set of possible edges \(\{E_{p}^{j}\}\) for the point \(P_{j} \) in the 3d object description. The next process would be to arrive at a set of definite edges \(\{E^{j}\}\) from this set.\\
For this purpose we use three corollaries, mentioned below:
\textit{
\begin{itemize}
\item If \space \(\exists T_{i} = [A_{T_{j}},A_{T_{k}}] \in \mathcal{T} \), such that there is a possible connecting edge \(E_{m}'' = (F_{l},F_{n}) \in \mathcal{E_{F}} \hspace{2pt} s.t. \hspace{2pt} A_{F_{j}} \in F_{l}, A_{F_{k}} \in F_{n}\) or vice-versa, then there must necessarily be an edge between the points \(P_{j}\) and \(P_{k}\), the corresponding points in the 3D description. Similar argument holds for the tuple \space \(\exists F_{i} = [A_{F_{j}},A_{F_{k}}] \in \mathcal{F} \). 
\textbf{Corollary I}
\item If, in the set \(\{E_{p}^{j}\}\), there exist more than one points such that, their 3D correspondences and the point \(P_{j}\) are collinear then \(P_{j}\) can only be connected to the next nearest point on this collinear line. \\
Hence, if the points other than \(P_{j}\) are on the same side of \(P_{j}\), then an edge shall exist only between \(P_{j}\) and the nearest point. \\
However, if these other points are on different sides of \(P_{j}\), then there can be an edge only between \(P_{j}\) and either of the two nearest points on each side. The other remaining points are discarded from the set \(\{E_{p}^{j}\} \). \textbf{Corollary II}
\item If \space \(\exists \) points \(P_{i}\) and \(P_{j}\), such that we have derived a definite edge between the two, and let there be a third point \(P_{k}\) such that \(P_{i} \in \{E_{p}^{k}\} \hspace{2pt} \& \hspace{2pt} P_{j} \in \{E_{p}^{k}\} \), then there can be a definite edge between \(P_{k}\) and one of \(\{P_{i}, P_{j}\} \). Note: The above two corollaries super-seed this one. \textbf{Corollary III}
\end{itemize}
}
We follow the above \textbf{Routine G} for every vertex \(P_{i}\), and this routine is iterated until there is no change in any of the sets of definite edges \(\{E_{p}^{i}\} \hspace{2pt} for \hspace{2pt} P_{i} \in \mathcal{V} \), the set of vertices in the 3D model description.\\
Hence we arrive at a simple wireframe model of the 3d object.\\
We delivered the set of vertices and set of edges which represent the 3d object.\\
This concludes our mathematical model of 3D to 2D as well as 2D to 3D reconstruction. 

\newpage
\section*{References}
\addcontentsline{toc}{section}{References}
\begin{enumerate}
\item  \textbf{Ray Casting Algorithm} \\
One simple way of finding whether the point is inside or outside a simple polygon is to test how many times a ray, starting from the point and going in any fixed direction, intersects the edges of the polygon. If the point is on the outside of the polygon the ray will intersect its edge an even number of times. If the point is on the inside of the polygon then it will intersect the edge an odd number of times.\\
We used  \href{https://en.wikipedia.org/wiki/Point_in_polygon}{Point in polygon} for learning about this algorithm.
\end{enumerate}



\end{document}
